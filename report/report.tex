\documentclass[10pt,twocolumn,a4paper]{article}
\usepackage[latin1]{inputenc}
\usepackage{amsmath}
\usepackage{amsfonts}
\usepackage{amssymb}
\usepackage{tikz}
\usepackage{listings}
\usepackage[width=17.00cm]{geometry}
\usepackage{graphicx}
\graphicspath{{./images/}}
\newcommand{\argt}{\theta}



\title{\LARGE{\textbf{Proyecto 2021} \\
		
		
		
        Estad�sticas\; \; \; Ciencias de la Computaci\'on\\
	
	
	
	\textbf{Orientaciones metodol\'ogicas:\\} Fase II\\}

Estudiantes David Guaty, Rodrigo Pino y Adrian Portales}

\date{}
%\renewcommand{\baselinestretch}{1.4}
\renewcommand{\labelenumi}{\alph{enumi}.}

\newtheorem{eje}{Ejercicio}
\newcommand{\sen}{\mbox{sen \hspace{0.001cm}}}
\newcommand{\cis}{\hspace{0.5mm}\mbox{cis}\hspace{0.5mm}}
\newcommand{\real}{\mathbb{R}}
\newcommand{\complex}{\mathbb{C}}




\begin{document}
\maketitle
\setcounter{page}{1}


\section*{Ejercicios}
\begin{eje}
    Realice un estudio de sus datos usando las t�cnicas de regresi�n, reducci�n de dimensi�n y de
    ANOVA.
    \begin{enumerate}
        \item Escoja las variables a las cuales les aplicara cada t�cnica y explique por qu�.
        \item En las t�cnicas que lo requieran realice el an�lisis de los supuestos y explique si es v�lida la
        aplicaci�n de la t�cnica en esa variable..
    \end{enumerate}

\end{eje}



\section*{Objetivos}
\begin{itemize}
	\item TODO
\end{itemize}
 
\section*{Introducci\'on} 

TODO
		


\section*{T\'ecnicas de Regresi\'on}
TODO



\section*{Reducci�n de dimensi�n} 

TODO

\section*{ANOVA}

TODO
\section*{Conclusiones}


TODO

\section*{Contribuciones de cada integrante}
TODO

  
\end{document}
