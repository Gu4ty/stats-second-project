\documentclass[10pt,twocolumn,a4paper]{article}
\usepackage[latin1]{inputenc}
\usepackage{amsmath}
\usepackage{amsfonts}
\usepackage{amssymb}
\usepackage{tikz}
\usepackage{listings}
\usepackage[width=17.00cm]{geometry}
\usepackage{graphicx}
\usepackage{hyperref}
\hypersetup{colorlinks=true,urlcolor=magenta}
\graphicspath{{./images/}}
\newcommand{\argt}{\theta}



\title{\LARGE{\textbf{Proyecto 2021} \\
		
		
		
        Estad�sticas\; \; \; Ciencias de la Computaci\'on\\
	
	
	
	\textbf{Orientaciones metodol\'ogicas:\\} Fase II\\}

Estudiantes David Guaty, Rodrigo Pino y Adrian Portales}

\date{}
%\renewcommand{\baselinestretch}{1.4}
\renewcommand{\labelenumi}{\alph{enumi}.}

\newtheorem{eje}{Ejercicio}
\newcommand{\sen}{\mbox{sen \hspace{0.001cm}}}
\newcommand{\cis}{\hspace{0.5mm}\mbox{cis}\hspace{0.5mm}}
\newcommand{\real}{\mathbb{R}}
\newcommand{\complex}{\mathbb{C}}




\begin{document}
\maketitle
\setcounter{page}{1}


\section*{Ejercicios}
\begin{eje}
    Realice un estudio de sus datos usando las t�cnicas de regresi�n, reducci�n de dimensi�n y de
    ANOVA.
    \begin{enumerate}
        \item Escoja las variables a las cuales les aplicara cada t�cnica y explique por qu�.
        \item En las t�cnicas que lo requieran realice el an�lisis de los supuestos y explique si es v�lida la
        aplicaci�n de la t�cnica en esa variable.
    \end{enumerate}

\end{eje}

\section*{Objetivos}
	\item Aplicar las t\'ecnicas estudiadas en clases para hacer un an\'alisis de los datos sobre en Censo de Estados Unidos del a\~no 1994.
	\item Ganar experiencia en el trabajo con el lenguaje R.
\section*{Introducci\'on} 

El dataset con el que trabajamos es un Censo de los Estados Unidos del a\~no 1994. El mismo cuenta con alrededor de 32000 \textit{records} y 15 variables, las cuales van desde la edad y la raza hasta el nivel de educaci\'on y la ocupaci\'on. El objetivo principal
de este trabajo es usar dichos datos para aplicar t\'ecnicas estudiadas en clases como ANOVA, regresi�n lineal y reducci�n de dimensi�n que nos permitan obtener resultados para caracterizar las variables pertenecientes al conjunto de datos y poder hacer predicciones tambi\'en. Todo esto con la ayuda del lenguaje de programaci\'on R.

\section*{T\'ecnicas de Regresi\'on}
\textit{(c\'odigo referente al an\'alisis de regresi�n lineal en lm.R)}

Por las caracter\'isticas de las variables de los datos con los que estamos trabajando, poder predecir el \textit{income} de un
individuo a partir del resto de las variables resulta de gran inter\'es. Por este motivo decidimos hacer una an�lisis y tratar de
obtener un modelo de regresi�n lineal que nos permita hacer predicciones sobre los ingresos anuales de una persona a partir de  un
grupo de datos del sujeto.  

Como primer paso antes de realizar cualquier an\'alisi con los datos se eliminaron todas aquellas entradas que pose\'ian valores
faltantes, ya que al final la muestra era bastante grande y eliminar dichos \textit{records} no representaba una p\'erdida  
considerable de los datos. De este modo evit\'abamos perturbaciones en los c\'aculos y an\'alisis posteriores.

Antes de hacer la regresi�n lineal nos interesaba analizar si exist\'ia alg\'un tipo de relaci\'on entre los datos ( ya sea lineal ono ). La gran
mayor\'ia de las variables son categ\'oricas y cont\'abamos con un muestra bastante grande ( alrededor de 32000 ) por lo que decidimos
hacer pruebas de Chi-Cuadrado de Independencia entre pares de variables categ\'oricas, sobre  todo \textit{income} con algunas
otras variables. En el caso de la educaci\'on y el sexo se obtuvo que las variables no eran independendientes, lo cual tiene
cierto sentido, pues se sabe que un mayor grado de educaci\'on implica mayores salarios y por el lado del sexo es bien sabido 
que en muchos lugares las mujeres suelen ser discriminadas y tienen salarios m\'as bajos en comparaci\'on con los hombres en puestos
equivalentes. Para el resto de combinaciones que se probaron con \textit{income} no se pudo obtener un resultado conciso, pues 
se emit\'ia una advertencia por parte de R que dec\'ia que los resultados pod\'ian no ser correctos.

Una vez hecho este peque\~no an\'alisis pasamos a la confecci\'on de los modelos de regresi�n lineal. Como hab\'iamos mencinoado
anteriormente la variable que resulta de mayor inter\'es para predecir es el \textit{income}, por lo que se intent\'o obtener un modelo
donde dicha variable fuera la dependiente. Cuando un modelo de regresi�n lineal se est\'a confeccionando, la opini\'on de un
experto en el campo de estudio tiene much\'isimo peso sobre la selecci\'on de las variables. Dado que no somos expertos decidimos
usar un enfoque \textit{backward}, es decir, planteamos un modelo  con todas las variables. Un punto importante a se\~nalar aqu\'
es que muchas de las variables son categ\'oricas y para poder hacer la regresi�n lineal sobre dicho conjunto de datos, estas pasan
por una transformaci\'on donde se convierten en variables binarias, en dependencia del nivel de cada categor\'ia. Lamentablemente 
este modelo no tuvo buenos resultados como se muestra a continuaci\'on:

\begin{lstlisting}[language=R,breaklines=true]
Residual standard error: 0.3451 on 30086 degrees of freedom
Multiple R-squared:  0.3644,    Adjusted R-squared:  0.3629
F-statistic:   230 on 75 and 30086 DF,  p-value: < 2.2e-16
\end{lstlisting}

Como podemos apreciar el \textit{R-squared} es bastante bajo, por lo que incluso si este modelo cumpliera los supuestos no 
ser\'ia muy bueno ( est\'a muy por debajo de 0.70 ). Curiosamente el \textit{p-value} es  significativo, por lo que al menos
podemos decir que el modelo est\'a haciendo algo. Despu\'es de obtener dicho resultado, pasamos analizar los supuestos del modelo:

En el caso de la media y la suma de los errores, las mismas toman valores muy cercanos a 0. Sin embargo, al realizar el test de 
Breusch-Pagan  se obtuvo que no cumpl\'ia la homocedasticidad. Por lo que al no cumplir uno de los supuestos nuestro modelo dej\'o
de ser factible. 

El siguiente paso fue quitar variables, en dependencia de la signicaci\'on de cada una, un proceso bastante engorroso dado que
las variables categ\'oricas pasaban a convertirse en variables binarias en dependencia del nivel de cada categor\'ia. Dado que lo
anterior tambi\'en resultaba un poco complicado, en algunos pasos se decidi\'o usar un criterio intuitivo, y en ocasiones se
prob\'o elimiando una variable y despues volviendola a a\~nadir para eliminar otra en su lugar.

Siguiendo el criterio de la significaci\'on de cada variable se decidi\'o eliminar como primera variable \textit{native.country}.
El modelo resultante obtuvo el siguiente resultado:

\begin{lstlisting}[language=R, breaklines=true]
Residual standard error: 0.3452 on 30126 degrees of freedom
Multiple R-squared:  0.3634,    Adjusted R-squared:  0.3626
F-statistic: 491.3 on 35 and 30126 DF,  p-value: < 2.2e-16
\end{lstlisting}

Como se puede apreciar el \textit{R-squred} disminuy\'o un poco, pero no de manera significativa. Aunque, por supuesto, el mismo
todav\'ia nos indicaba la presencia de un modelo bastante malo, pero quiz\'a cumpl\'ia con los supuestos. Sin embargo, pasaba lo
mismo que el modelo inicial: no se cumpl\'ia la homocedasticidad.

En este punto decidimos seguir con el proceso de eliminaci\'on de variables. La siguiente variable fue \textit{rece}. Los 
resultados fueron los siguientes:

\begin{lstlisting}[language=R, breaklines=true]
Residual standard error: 0.3453 on 30130 degrees of freedom
Multiple R-squared:  0.363,     Adjusted R-squared:  0.3623
F-statistic: 553.8 on 31 and 30130 DF,  p-value: < 2.2e-16
\end{lstlisting}

Se puede ver que el \textit{R-squared} sigue disminuyendo, aunque no de manera significativa. Lamentablemente, este modelo tampoco
cumpl\'ia los supuestos: fallaba en el test de homocedasticidad. 

La b\'usqueda de un conjunto de variables independendientes continu\'o, pero no fuimos capaces de un encontrar un modelo \'util ( quiz\'a nuestra b\'usqueda no fue lo suficientemente exahustiva ),
pues en primer lugar la tendencia a la dismuci\'on del \textit{R-squared} segu\'ia y ya desde el principio el valor del mismo
nos indicaba que el modelo no era bueno.

Por el an�lisis hecho con la prueba de independencia de Chi-cuadrado, pensamos que s\'i podr\'ia ser posible encontrar un modelo para
predecir el \textit{income} de los individuos, sin embargo con los resultados obtenidos a trav\'es de la b\'usqueda del mismo a trav\'es de  regresi�n
lineal podemos concluir que si existe realmente dicha relaci\'on, la misma no debe ser lineal y ser\'ia necesario el uso de t\'ecnicas m\'as sofisticadas.

\section*{Reducci�n de dimensi�n} 
\subsection*{C\'odigo}
Para ejecutar el c\'odigo y observar las gr\'aficas de las cuales se dedujeron los siguientes an\'alisis ejecutar los scripts en "src/reduction\_techniques/". C\'odigos:
\begin{itemize}
    \item
        \textit{k-means(k-prototype)} est\'a implementado en kmeans.R
    \item
        \textit{Clasificaci\'on} est\'a implementado en class.R
    \item
        \textit{Componetes Prinicpales} est\'a implementado en pca.R
    
\end{itemize}


\subsection*{k-means(k-prototype)}

Se cuenta con un conjunto de datos compuesto mayoritariamente por variables categ\'oricas donde k-means al ser un m\'etodo que utiliza la distancia euclidiana no le es posible trabajar con datos categ\'oricos.

En 1995 Ralambondrainy present\'o una soluci\'on donde se convierten las variables categ\'oricas en varias variables binarias. Esto no es factible para nuestro set de datos pues aumentar\'ia la dimension de 15 a 108 columnas!

Otra soluci\'on es presentada por \href{www.cs.ust.hk/~qyang/Teaching/537/Papers/huang98extensions.pdf}{Huang} en 1997 donde se propone una extensi\'on de kmeans para variables categ\'oricas, kmodes, que utiliza las modas y las frecuencias de los datos cualitativos.
Este algoritmo es poco pr\'actico en la realidad pues las variables rara vez se muestran s\'olo como categ\'oricas. 
Huang abord\'o este problema presentando kprototype que utiliza kmodes cuando trabaja con variables categ\'oricas y kmeans en otro caso. Cuandos son dos variables cuantitativas se utiliza la distancia euclididana y cuando es una mezca se utiliza la distancia de gowers.

Nuestro set de datos se analizar\'a utilizando la implementaci\'on de kprototype en la biblioteca \textit{clustMixType}.

Antes de continuar el an\'alisis se debe tener claro ciertas cosas sobre los datos:
\begin{itemize}
    \item \textit{bachelor} en E.E.U.U es equivalente a licenciado.
    \item \textit{highschool} significa haber terminado el pre universitario o el bachillerato.
    \item \textit{some-college} es cuando se empieza la universidad, pero no se termina.
    \item \textit{own-child} es el t\'ermino legal para los ni\~nos/j\'ovenes que viven con sus familias.
\end{itemize}

Para detectar el n\'umero de cl\'usters necesarios se utiliz\'o el m\'etodo del codo, la figura no obstante no deja bien claro cual es el n\'umero de cl\'usters \'optimo, se intuye que debe ser un n\'umero entre 5 y 9. Vale mencionar que \textit{clustMixType} viene con m\'etodos implementados para calcular dicho n\'umero, no obstante la \textit{ram} de nuestras PCs result\'o insuficientes y no se pudieron utilizar.

\subsubsection*{An\'alisis de cl\'usters}
Se decidi\'o realizar el an\'alisis con 9 cl\'usters, se muestran las caracter\'isticas principales de cada uno:

El primer grupo se encuentra conformado por mujeres, de las cuales hay negras y blancas, con una edad superior a los 45 a\~nos que trabajan principalmente en el sector privado. El nivel de educaci\'on ronda sobre pre universitario o \textit{some-college}  La mayor\'ia de estas personas no se encuentran en ning\'un tipo de relaci\'on debido a separaciones o enviudamientos. Trabajan ofreciendo servicios no catalogados por los datos. Es el grupo poblacional (junto con el sexto) que menos horas al d\'ia trabaja y ganan menos de 50k.

El segundo grupo esta compuesto mayoritariamente por hombres blancos que trabajan en el sector privado y para el gobierno. Son un grupo de personas que por regla general tiene m\'as de un t\'itulo universitario (m\'asters y doctorados) o cursaron la escuela profesional. La mayor\'ia de sus integrantes se encuentran casados y realizan trabajos como profesionales especializados. Trabajan entre 40 y 50 horas semanales con un salario que supera los 50k.

El tercer grupo esta integrado por hombres blancos casados entre los 30 y 40 a\~nos que trabajan en el sector privado y tienen t\'itulo de bachiller, algunos participaron en la universidad, pero no la terminaron. Trabajan de 40 a 50 horas semanales ofreciendo servicios de reparaci\'on y transporte, obteniendo un salario anual menor que 50k.

El cuarto grupo esta compuesto por mujeres blancas entre los 30 y 40 a\~nos con nivel pre universitario, \textit{some-collge} y licenciados. Se encuentran fuera de una relaci\'on debido al divorcio o nunca se han casado. Estas personas trabajan 40 horas semanales haciendo trabajos administrativos o clericales (ordenar emails, escribir documentos, responder el tel\'efono etc.) con un salario menor a 50k.

El quinto grupo consiste en hombres entre los 35 y 45 a\~nos de edad, una porci\'on importante es de nacionalidad mexicana. Son las personas de menor nivel educativo de la muestra, el nivel m\'as alto obtenido es el pre universitario. Trabajan ofreciendo servicios diversos o en ventas durante 40 horas a la semana cobrando anualmente menos de 50k.

El sexto grupo esta constituido por los hombres y mujeres m\'as j\'ovenes de la muestra, entre los 20 y 25 a\~nos. El nivel de educaci\'on predominante es el bachillerato y \textit{some-college}. Muchos de ellos aun viven con su familia. Realizan todo tipo de trabajos en el sector privado lucrando anualmente menos de 50k. Su horario laboral est\'a entre los menores (de 20 a 40 horas semanales).

El s\'eptimo grupo son mujeres y hombres tanto blancos y negros que terminaron el pre universitario, pero no la universidad con una edad entre los 25 y 35 a\~nos. Trabajan en el sector privado distribuidos en todas las categor\'ias, con mayor incidencia en \textit{craft-repair} y \textit{transport-moving}. Su jornada laboral como promedio es de 40 a 45 horas semanales con un cobro anual menor que 50k.

El ocatvo grupo esta compuesto por hombres blancos mayores de 45 a\~nos que se encuentran en una relaci\'on,  graduados del pre universitario o que han iniciado pero no finalizado la universidad. Trabajan mayoritariamente en el sector privado ofreciendo servicios de  \textit{craft-repair}, \textit{sales} y \textit{transport-moving} de 40 a 45 horas semanales lucrando por ello un salario mayor a los 50k.

El noveno y \'ultimo grupo esta representado por hombres blancos casados entre los 30 y 40 a\~nos de edad. Casi todos los integrantes de este grupo poseen un t\'itulo universitario, trabajan entre 40 y 50 horas semanales como managers ejecutivos con un salario anual superior a los 50k.

\subsubsection*{An\'alisis de los datos}
Los grupos formados evidencian relaciones muy interesantes en los datos:
\begin{itemize}
    \item 
        La mayor\'ia de los hombres adultos entrevistados suelen estar ya en una relaci\'on, mientras que las mujeres adultas est\'an en su mayor\'ia solteras debido a divorcios, separacions o enviudamientos. Surge la duda de por que ocurre esto, acaso las mujeres trabajadores suelen ser m\'as independientes, emancipadas? O es que se est\'a ante unos datos no representativos de la poblaci\'on en este aspecto.

    \item El primer, tercer y octavo grupo poseen un nivel educativo  muy similar, no obstante el \'ultimo es el \'unico con un salario mayor que 50k. \\
La diferencia m\'as notable entre el tercer y octavo grupo es la edad de los integrantes, son muy similares en los dem\'as aspectos. Podemos decir entonces que la edad es un factor importante en la ganancia de una persona.\\
El primer y octavo grupo se encuentran en un rango parecido en edad, sin embargo son muy diferentes en el resto de los aspectos: trabajan en cosas distintas, una cantidad de horas distintas y est\'an compuestos mayormente por sexos opuestos. Se intuye que lo que m\'as influye es la ocupaci\'on y las horas de trabajo.\\
Se cree que el sexo tambi\'en influye de manera indirecta pu\'es este condiciona la ocupaci\'on de una persona. Por ejemplo, el cuarto grupo constituido mayormente por mujeres con un alto nivel de educaci\'on se centran en trabajos administrativos y clericales. No hay concentraci\'on grande de mujeres en trabajos manuales o ejecutivos.

\item El segundo y noveno grupo presentan el m\'as alto nivel acad\'emico de la muestra. Tambi\'en tienen cargos ejecutivos o especializados, son mayores de 40 a\~nos y trabajan 40 horas o m\'as al d\'ia con un salario superior a los 50k. 
Se puede decir entonces que las variables que m\'as influyen sobre el salario es edad, ocupaci\'on, horas de trabajo y nivel de estudio.

\end{itemize}

\subsection*{Clasificaci\'on}
Se construir\'a un \'arbol de clasificaci\'on para predecir el \textit{income} de una persona. Se espera que el \'arbol busque las variables m\'as relevantes y se pueda reducir las dimensi\'on del conjunto de datos a las seleccionadas por el \'arbol.

El \'arbol se puede obtener cuando se ejecuta el c\'odigo respectivo a esta secci\'on o visualizar directamente en "src/reduction\_techniques/images/tree.png"

\subsubsection*{An\'alisis del \'arbol}
Cuando se observa el \'arbol obtenido las variables principales para el algoritmo son el tipo de \textit{relationship}, \textit{educacation} y \textit{capital.gain}. Esto difiere de lo analizado en \textbf{k-means (k-prototype)}.

En el primer nivel del \'arbol se pregunta el tipo de relaci\'on que presenta la persona. Eso ocurre pues los datos est\'an sesgados en este sentido. Por el an\'alisis de la secci\'on anterior se conoce que las personas que cumplen la relaciones especificadas son por regla general j\'ovenes y mujeres, el sector poblacional que gana menos, en cambio, el conjunto de los hombres adultos que contienen dentro de s\'i a la mayor\'ia de las personas que ganas m\'as de 50k est\'an casi todos en una relaci\'on.

En el segundo nivel pregunta por el nivel de educaci\'on. Si es bajo (no es licenciado, ni m\'aster ni doctor, ni fue a la escuela profesional) entonces debe ganar mas de 50k. Esto coincide  con nuestro an\'alisis en la secci\'on anterior.

En el tercer y \'ultimo  nivel separa por el \textit{capital.gain}. Esta variable no se considera buena pues los valores est\'an muy centrados en cero, con algunos puntos aberrantes.

Cuando calculamos las m\'etricas para predecir la eficacia del algoritmo se obtienen buenas puntuaciones de  accuracy, recall y precission. Aunque el accuracy esta sobre el ~83\% debemos tener en cuanta que los datos se encuentran sesgados en lo que se refiere al \textit{income}, solo 1/3 de la muestra cumple esto.

Si calculamos tambi\'en el specifity (un recall para los \textit{True Negatives}) o observamos la matriz de confusi\'on se evidencia que el CART falla en predecir correctamente a las personas que ganan m\'as de 50k la mitad de las veces. Los \textit{True Negatives} y \textit{False Negatives} tienen aproximadamente la misma cantidad. No obstante como estamos frente a una muestra donde predominan las personas que ganan menos de 50k y estos se predicen en su mayor\'ia correctamente se oculta el hecho de que predecir cuando ganan m\'as sea equivalente a tirar una moneda.

\subsection*{Componentes Principales}
Para intentar realizar un an\'alisis de componentes principales fue necesario convertir las variables categ\'oricas a variables binarias. 
Las variables categ\'oricas no se puede llevar directamente a n\'umeros de la forma que cada uno representa una categor\'ia pues el resultado de realizar PCA no tuviera sentido. Por ejemplo, si una componente resulta con \textit{marital.status} = 0.8 no es posible interpretarlo.

Dividiendo las variables categ\'oricas en binarias, aumentan el n\'umero hasta 108. Es posible reducirlas a una menor cantidad agrup\'andolas en subgrupos semejantes. Por ejemplo \textit{native.country} que se transforma 41 variables binarias se pueda agrupar en continentes. Con las dem\'as variables categ\'oricas se puede realizar algo similar.

Reduciendo la cantidad de categor\'ias de cada variable se puede reducir la dimensi\'on pero no lo suficiente para luego aplicar PCA y extraer grupos representativos de la muestra.

La utilizaci\'on de esta t\'ecnica conlleva a un incremento en la dimensi\'on de la muestra.


\section*{ANOVA}

\textit{(c\'odigo referente al an\'alisis anova en anova.R)}

En esta secci\'on se buscar\'an diferencias significativas entre grupos de una variable. 

Por ejemplo, nos podr\'iamos preguntar cual \textit{workclass} tiene mayor \textit{income}, o cual de las \textit{occupation} tienen un mayor promedio de edad. Estos an\'alisis nos sirvir\'ian para diferenciar caracter\'isticas entre grupos de personas.

Una de las variables m\'as importantes a analizar es el \textit{income}, ya que es la variable que se quiere predecir dados los dem\'as datos. Pero tenemos que la variable \textit{income} es categ\'orica(con categor\'ias $\le 50K$ y $ >50K$). Por lo que convertimos la variable \textit{income} a una variable num\'erica donde se le asigna el valor 1 a la categor\'ia  $\le 50K$ y el valor 2 a la categor\'ia $ >50K$. Con esta asignaci\'on es posible realizar un anova en el cual la variable dependiente sea el \textit{income}. Esto tiene sentido porque no nos interesa el significado exacto de la media de la variable num\'erica \textit{income}, sino la diferencias significativas entre las medias de los distintos grupos analizados. As\'i por ejemplo podemos analizar cual \textit{workclass} posee un mayor \textit{income}.

Para la realizaci\'on de los an\'alisis anova tomando como variable dependiente el \textit{income} se escogieron las variables: \textit{education}, \textit{occupation}. En su mayor parte, un mayor nivel de \textit{education} se correlaciona con un mayor porcentaje de individuos con $>50k$ de \textit{income}. El salario de una persona depende fuertemente de su profesi\'on, existen profesiones que tienen un mayor porcentaje de individuos con $>50K$ de \textit{income}. La variable \textit{sex} es otro buen predictor del \textit{income}, pero la variable \textit{sex} al tener solo dos categor\'ias ser\'ia mejor realizar un an\'alisis mediante t-student y no anova, ya que solo que comparar\'ian dos medias, parecido a como se hizo en la primera fase del proyecto. 

Las variables escogidas son categ\'oricas y se puede analizar que categor\'ia presenta una diferencia significativa respecto al \textit{income}. 

Para realizar los an\'alisis anova en R, se implement\'o una funci\'on auxiliar que puede imprimir dos cosas: de cumplirse los supuestos del modelo se imprime el \textit{summary} del resultado de la funci\'on \textit{aov}, en caso de no cumplirse los supuestos se imprime el mensaje "assumptions not fulfilled", dando a entender que el modelo no funciona.

A continuaci\'on se explica por partes el c\'odigo de la funci\'on:

\begin{lstlisting}[language=R,title= Funcion do\_anova Parte 1, breaklines=true]
do_anova <- function(independent, dependent,name_of_independent, name_of_dependent){
independent <- sample(independent, 1000)
dependent <- sample(dependent, 1000)

anova_data <- data.frame(independent, dependent)
anova_data <- anova_data[order(anova_data$independent),]
plot(dependent ~ independent, data = anova_data, ylab = name_of_dependent, xlab= name_of_independent)
\end{lstlisting}

En esta parte se leen dos vectores: independent y dependent. Se toma una muestra de 1000 elementos de ambos. Se conforma un data frame y adem\'as se ordena el data frame por la variable independendiente categ\'orica, as\'i el data frame queda estructurado como fue visto en conferencia para la correcta utilizaci\'on del anova. Adem\'as, se grafican las distintas categor\'ias en un gr\'afico de caja para analizar gr\'aficamente si existen diferencias. 

\begin{lstlisting}[language=R,title= Funcion do\_anova Parte 2,  breaklines=true]
result <- aov(dependent ~ independent, data = anova_data)
res <- result$residuals
\end{lstlisting}

En esta parte se lleva a cabo el anova, se guardan en la variable \textit{res} los residuos para comprobar los supuestos.

\begin{lstlisting}[language=R,title= Funcion do\_anova Parte 3,  breaklines=true]

is_model_ok = TRUE
stest <- shapiro.test(res)
if(stest$p.value < 0.05){
	is_model_ok = FALSE
}

btest <- bartlett.test(res, anova_data$independent)
if(btest$p.value < 0.05){
	is_model_ok = FALSE
}
dtest <- dwtest(result)
if(dtest$p.value < 0.05){
	is_model_ok = FALSE
}
if(is_model_ok){
	summary(result)
}
else{
	print("assumptions not fulfilled")
}
\end{lstlisting}
En esta parte se hacen todas las pruebas para los supuestos. Si alguna prueba falla entonces no se cumplen los supuestos del modelo y se imprime el mensaje "assumptions not fulfilled". En caso de cumplirse los supuestos se imprime un resumen del anova.

\textbf{Income $\sim$ education}
\begin{lstlisting}[language=R, breaklines=true]
> do_anova(data$education, data$income, "education", "income")
[1] "residuals do not have a normal distribution"
[1] "residuals are not homogeneous"
[1] "assumptions not fulfilled"
\end{lstlisting}

Por lo que el modelo no cumple los supuestos.

\textbf{Income $\sim$ occupation}
\begin{lstlisting}[language=R, breaklines=true]
> do_anova(data$occupation, data$income, "occupation", "income")
[1] "residuals do not have a normal distribution"
[1] "assumptions not fulfilled"
\end{lstlisting}

Por lo que el modelo no cumple los supuestos.

En las variables escogidas no se cumple el modelo. Veamos otras variables.

\textbf{Income $\sim$ workclass}
\begin{lstlisting}[language=R, breaklines=true]
> do_anova(data$workclass, data$age, "workclass", "age")
[1] "residuals do not have a normal distribution"
[1] "assumptions not fulfilled"
\end{lstlisting}

Por lo que el modelo no cumple los supuestos.

\textbf{Income $\sim$ marital.status}
\begin{lstlisting}[language=R, breaklines=true]
> do_anova(data$marital.status, data$income, "marital.status", "income")
[1] "residuals do not have a normal distribution"
[1] "assumptions not fulfilled"
\end{lstlisting}

Por lo que el modelo no cumple los supuestos.

Por lo que no se puede aplicar Anova, al menos con la variable \textit{income}.

Pasemos a analizar otros pares de variables como \textit{workclass} y \textit{age}, podr\'iamos ver cual es el grupo de trabajadores con  mayor o menor edad.

\textbf{Age $\sim$ workclass}
\begin{lstlisting}[language=R, breaklines=true]
> do_anova(data$workclass, data$age, "workclass", "age")
[1] "residuals do not have a normal distribution"
[1] "assumptions not fulfilled"
\end{lstlisting}

Por lo que el modelo no cumple los supuestos.

Tratemos otro modelo con age. Seleccionamos la variable \textit{occupation}. Puede ser interesante saber la ocupaci\'on que tiene m\'as empleados de una edad mayor.

\textbf{Age $\sim$ occupation}
\begin{lstlisting}[language=R, breaklines=true]
> do_anova(data$occupation, data$age, "occupation", "age")
[1] "residuals do not have a normal distribution"
[1] "assumptions not fulfilled"

\end{lstlisting}

Por lo que el modelo no cumple los supuestos.

Tratemos de analizar si existe alguna diferencia entre el promedio de la ganancia capital entre los individuos de diferentes ocupaciones.

\textbf{capital-gain $\sim$ occupation}
\begin{lstlisting}[language=R, breaklines=true]
> do_anova(data$occupation, data$capital.gain, "occupation", "capital.gain")
[1] "residuals do not have a normal distribution"
[1] "residuals are not homogeneous"
[1] "assumptions not fulfilled"
\end{lstlisting}

Por lo que el modelo no cumple los supuestos.

Otro caso de inter\'es es saber si existe diferencia significativa entre el promedio de las horas de trabajo semanales entre los individuos de diferentes profesiones.
 
\textbf{hours-per-week $\sim$ occupation}
\begin{lstlisting}[language=R, breaklines=true]
> do_anova(data$occupation, data$hours.per.week, "occupation", "capital.gain")
[1] "residuals do not have a normal distribution"
[1] "residuals are not homogeneous"
[1] "assumptions not fulfilled"
 \end{lstlisting}
 
 Por lo que el modelo no cumple los supuestos.
 
 Para no hacer m\'as extensa la secci\'on, se mencionar\'a que tambi\'en se comprob\'o el anova con otros pares de variables. Digase: native-countre$\sim$education.num, native-countre$\sim$hours-per-week, race$\sim$hours-per-week, entre otras. En todas ellas los modelos no cumpl\'ian los supuestos. 
 
 Por lo que se puede afirmar con cierta seguridad que no se puede aplicar t\'ecnicas de anova a los datos. Al menos no se encontr\'o una forma de aplicarlas :-(. Los investigadores est\'an tristes.  

\section*{Conclusiones}

EL an\'alisis de los datos muestran que estamos frente a datos con muy poca correlaci\'on y sesgados. Las t\'ecnicas de car\'acter l\'ineal surten poco afecto, adem\'as de que impide que se cumplan supuestos. Regresi\'on l\'ineal y ANOVA no fueron efectivas. Tambi\'en las t\'ecnicas estudiadas en clases se enfoca principalmente en valores cualitativos y nuestro set es en su mayori\'ia cualitativo, eso requiri\'o un aumento de dimensi\'on en los datos y por ende complicaciones adicionales. 

\section*{Contribuciones de cada integrante}

 Para la realizaci\'on del proyecto se sigui\'o la siguiente estrategia:

 Cada integrante escogi\'o un tipo de t\'ecina para aplicar ( reducci�n de dimensi�n, regresi�n lineal o ANOVA ). Luego
 de aplicar dicha t\'ecnica, el integrante le explica le procedimiento seguido y los resultados obtenidos al resto del equipo, y estos 
 tratan de encontrarle alg\'un fallo en el c\'odigo o en alguna explicaci\'on del informe que no est\'e muy clara.
 
 Siguiendo esta estrategia las tareas se dividieron de la siguiente forma: Adri\'an Rodr\'iguez Portales regresi�n lineal, David Guaty ANOVA, Rodrigo Pino reducci�n de dimensi�n.
 
\end{document}
